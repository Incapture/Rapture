This call creates a named lock manager in \Rapture. The configuration string is
in fact a complex instruction written in a lock domain specific language (DSL) that is used to define the
capabilities and underlying implementation of the repository.

The typical configuration string for a locking provider backed by MongoDB is reproduced below:

\begin{Verbatim}
LOCK {} USING MONGODB { prefix = 'test' }
\end{Verbatim}

The general form of the configuration is:

\begin{Verbatim}
LOCK { }
     USING [underlying implementation] { [ config ]}
     [ ON [ instance] ]
\end{Verbatim}

The second part of the configuration string defines the underlying implementation and its configuration. In
most cases the configuration associated with the implementation has a \verb+prefix+ parameter that is used to
define a table or a collection or a prefix for such entities in the underlying storage. The underlying implementation
defines what lower level software is used to host the data managed by \Rapture. The following table shows the current
implementations:

\begin{table}[h]
\begin{center}
\begin{tabular}{r l p{8cm}}
  Keyword & Underlying & Configuration \\
  \hline
  MONGODB & MongoDb & The prefix parameter defines the name of the collections used by this repository. To avoid
  conflict this is usually a function of the name of the \Rapture repository. \\
  CASSANDRA & Cassandra & The prefix parameter defines the name of the collections used by this repository. To avoid
  conflict this is usually a function of the name of the \Rapture repository. \\
\end{tabular}
\end{center}
\end{table}
