The \verb+createIdGen+ call is used to define an ID generator source.

The typical configuration string for a locking provider backed by MongoDB is reproduced below:

\begin{Verbatim}
IDGEN { initial="1",
        base="36",
        length="8",
        prefix="ID" } USING MONGODB { prefix = 'test' }
\end{Verbatim}

The general form of the configuration is:

\begin{Verbatim}
IDGEN { config }
     USING [underlying implementation] { [ config ]}
     [ ON [ instance] ]
\end{Verbatim}

The first part of the configuration defines the way in which the provider will generate ids. The fields
and the meanings are reproduced below:

\begin{table}[h]
\begin{center}
\begin{tabular}{r l p{8cm}}
  Field & Example & Description \\
  \hline
  initial & 1 & The id number of the first id to be produced by this provider. \\
  base & 36 & The base of the number system to be used by the provider. Base 36 implies letters and digits. \\
  length & 8 & The width of the id. The id will be padded with zeroes to this length. \\
  prefix & ID & The id will be prefixed with this string. \\
\end{tabular}
\end{center}
\end{table}

As an example, this producer configuration would produce the following ids:

\begin{Verbatim}
  ID00000199
  ID0000019A
  ID0000019B
\end{Verbatim}

The second part of the configuration string defines the underlying implementation and its configuration. In
most cases the configuration associated with the implementation has a \verb+prefix+ parameter that is used to
define a table or a collection or a prefix for such entities in the underlying storage. The underlying implementation
defines what lower level software is used to host the data managed by \Rapture. The following table shows the current
implementations:

\begin{table}[h]
\begin{center}
\begin{tabular}{r l p{8cm}}
  Keyword & Underlying & Configuration \\
  \hline
  MONGODB & MongoDb & The prefix parameter defines the name of the collections used by this repository. To avoid
  conflict this is usually a function of the name of the \Rapture repository. \\
  CASSANDRA & Cassandra & The prefix parameter defines the name of the collections used by this repository. To avoid
  conflict this is usually a function of the name of the \Rapture repository. \\
\end{tabular}
\end{center}
\end{table}
