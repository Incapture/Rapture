The \verb+getCursor+ call executes a query against a table that will return a cursor that can be paged through. Such an
object can then be used with the \verb+next+ and \verb+previous+ calls and closed with \verb+closeCursor+.

\begin{table}[H]
\begin{center}
\begin{tabular}{r p{10cm}}
  Field & Purpose \\
  \hline
  table & This is simply a full tableUris used in the select. \\
  columnNames & This lists the column names that will be returned as part of this call. The column names should be prefixed with the table name. \\
  where & A suitable WHERE clause if required. \\
  order & Which columns should be used to order the results. \\
  ascending & Whether the order should be ascending (true) or descending (false). \\
  limit & The maximum number of rows that will be returned. \\
\end{tabular}
\end{center}
\end{table}

\begin{Verbatim}
  table = '//test/one';
  columns = [ 'firstname', 'lastname'];
  where = "lastname LIKE 'Mo%'";
  order = [ 'firstname' ];
  ascending = true;
  limit = 100;

  c = #structured.getCursor(table,
      columns, where, order, ascending, limit);
  v = #structure.next(table, c, 10);
  #structure.closeCursor(table, c);
  println(v);
\end{Verbatim}
