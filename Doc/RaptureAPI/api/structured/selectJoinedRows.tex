The \verb+selectJoinedRows+ call is used to perform a select with join across a number of tables in a
structured repository (database).

The parameters require some explanation, the following table and the example code serve this purpose.

\begin{table}[H]
\begin{center}
\begin{tabular}{r p{10cm}}
  Field & Purpose \\
  \hline
  tableUris & This is simply a list of the full tableUris used in the join. \\
  columnNames & This lists the column names that will be returned as part of this call. The column names should be prefixed with the table name. \\
  from & The from clause, which will include the JOIN clause. (See the example). \\
  where & A suitable WHERE clause if required. \\
  order & Which columns should be used to order the results. \\
  ascending & Whether the order should be ascending (true) or descending (false). \\
  limit & The maximum number of rows that will be returned. \\
\end{tabular}
\end{center}
\end{table}

\begin{Verbatim}
  tables = [ '//test/one', '//test/two'];
  columns = [ 'one.firstname', 'two.city'];
  from = 'test.one INNER JOIN test.two ON one.id=two.id';
  where = '';
  order = [ 'one.firstname' ];
  ascending = true;
  limit = 10;

  results = #structured.selectJoinedRows(tables,
    columns, from, where, order, ascending, limit);
  println(json(results));
\end{Verbatim}
