
The document API for \Rapture is often abbreviated to \emph{Doc}. The API is used
to manipulate the presence and the content of document repositories in \Rapture.

In the abstract a document repository in \Rapture is a key/value store with optional
enhancements. The key in \Rapture corresponds to a URI for the document and where the
context is not obvious the scheme of the uri is \verb+document://+. In all document
API calls this scheme may be omitted.

Document repositories in \Rapture are backed by concrete data storage systems. When
you define a repository in \Rapture you provide a configuration string that is used
by \Rapture to route your request to a low level driver that interacts with the
underlying system. The format of this configuration string will be described in
the API call for creating a repository.

Document repositories can also be versioned. When you update a document in a
versioned repository the previous history of that document is preserved. In fact you can
qualify the URI of a document with the @ symbol and a version number to retrieve
previous versions of a document. Omitting the @ symbol will always retrieve the
latest version of a document.

Documents in repositories can also have metadata associated with them. \Rapture
automatically maintains some of this metadata - the time the document was created, the
user that created it. But a developer can use metadata update calls to add their
own attributes to documents in \Rapture.

The URI of a document in a repository implies a folder-like structure with the
forward slash delineating these folders. There are document API calls to treat a
document repository like a file system -- these are useful when constructing
browsable user interfaces to a repository.

\subsection{Methods}
