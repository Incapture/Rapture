
The blob API for \Rapture is used to manipulate the presence and the content of
blob repositories in \Rapture.

A blob in this context is an arbitrary large piece of binary data that the underlying
\Rapture system does not have to process -- in effect a blob repository is simply a store
of content. Each blob is keyed by a \Rapture uri and where the context is not obvious the scheme
of the uri is \verb+blob://+. In all Blob API calls this scheme may be omitted.

Blob repositories in \Rapture are backed by concrete data storage systems. When
you define a blob repository in \Rapture you provide a configuration string that is used
by \Rapture to route your request to a low level driver that interacts with the
underlying system. The format of this configuration string will be described in
the API call for creating a blob repository.

Blobs in repositories can also have metadata associated with them. \Rapture
automatically maintains some of this metadata - the time the document was created, the
user that created it and the size of the content. \Rapture also attempts to guess (using the
"extension" of the uri name) the mime-type of the content but this can also be overridden. The
mime type is often used by user interfaces as a way of attempting to display the content in a
meaningful way.

The URI of a blob in a repository implies a folder-like structure with the
forward slash delineating these folders. There are blob API calls to treat a
blob repository like a file system -- these are useful when constructing
browsable user interfaces to a repository.

In the \Rapture environment blobs can also be manipulated from a servlet that works in a more
RESTful manner for retrieving the content and POSTing the content. This technique is much more
appropriate for client side calls.

\subsection{Methods}
