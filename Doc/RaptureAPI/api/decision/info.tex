The Decision API in \Rapture manages all aspects of Workflows. A \emph{workflow} in \Rapture
is a sequence of steps that describe a formal process that can be executed from a \Rapture
server. The steps can split, branch, rejoin and the flow of execution can be decided during
the execution of a workflow.

When running the status and context for a workflow is captured in a \emph{work order} (WorkOrder).

Workflows are typically attached to general events or schedules. They can be embeded in other
workflows and can also be invoked directly from the API.

WorkOrders can be paused, cancelled and resumed. The output from a WorkOrder is typically captured
in a system blob repository for later review.

A step in a workflow can be implemented by running a \Reflex script or through a developer supplied
Java class that conforms to a specific interface.

An addon to \Rapture from Incapture provides a custom step and execution environment for allowing
external processes (such as Python scripts) to be run as part of a workflow step.

\subsection{Workflows}

A workflow in \Rapture is referenced using a uri in a similar way to all other entities in the system. It can also
have a description. The main body of a workflow is a set of \emph{Steps} that form its definition.

\subsubsection{Steps}

A step in a workflow has a name and a description. The configuration for a step also has an implementation
section that defines either a class name of the Java code that will execute when the step is run or a \Reflex
script.

The return value from the execution of a step (either through \Reflex or Java) is a string. This string is used
by the workflow engine to determine what the next step should be. The mapping that enables this is called the
\emph{transitions} of a step.

One step is identified in a workflow as the "starting step".

\subsubsection{Transitions}

Transitions manage how the execution of a workflow flows between steps. A transition is simply a mapping between
the return value of a step execution and the next step associated with that return value. The transitions of steps
in a workflow describe the directed graph that defines the workflow.

In transitions there are some keyword based transitions that have special meanings. They could also be called in-built
steps. The step "\$RETURN:xxx" implies that this workflow should end with the status code of "xxx". The status code is
important as this could form the step transition code for an outer workflow.

\subsubsection{Views and Contexts}

As part of a workflow execution in a work order there is the concept of a view context. These are variables
that are associated with the execution of a workflow or the execution of an individual step. (Some steps can be executed in
parallel and each parallel execution may wish to have a slightly different context). There are calls in the decision api
that can manipulate these context values and the step and workflow definition in the workflow can set default (or fixed) values
of these variables.

Context variables can also be set as part of workflow execution -- either in a general way through the invocation of the call
to create a work order from a workflow or from other mechanisms (e.g. events or schedules). Views and contexts are important
techniques for passing on transient information very specific to the execution of a work order between the steps.

\subsubsection{Categories}

The code for a given step execution could be very specific to the server it is running on -- perhaps it is looking at
a specific folder on a hard drive, or perhaps it runs a Python scripts which requires the underlying server to have a set
of modules pre-installed.

The category facility of a step (and in a general sense a workflow) can be used to determine what \Rapture servers a
step can be executed on. A step can be associated with a category (a server group) and a \Rapture server can be configured
to be part of a given category (or many categories). Only servers that are members of a category can execute steps for that
category.

\subsection{Work Orders}

Work Orders are instances of Workflows (to use a programming analogy). A work order references its workflow but is a unique
instance of the workflow -- all variables, execution, context and status are associated with the work order and not the workflow.

Workflows define what is to be executed and the flow between steps during execution. A work order captures the live state of an
instance of a workflow.

\subsection{Locks}

The configuration of a workflow can restrict the number of running instances can be in a \Rapture system at any one time. Some
workflows may have resource dependencies which imply that only one (or a small number) of active runs should be in progress. This
is known as the lock or semaphore configuration of a workflow.

\subsection{Methods}
