The \verb+createSeriesRepo+ is used to create a new series repository in \Rapture. The parameters to the call
look straightforward -- simply the name of the new repository and a configuration string. The configuration string is
in fact a complex instruction written in a repository domain specific language (DSL) that is used to define the
capabilities and underlying implementation of the repository.

The typical configuration string for a series repository backed by MongoDB is reproduced below:

\begin{verbatim}
SREP {} USING MONGODB { prefix = 'test' }
\end{verbatim}

The general form of the configuration is:

\begin{verbatim}
SREP { [ series repo config] }
     USING [underlying implementation] { [ config ]}
     [ ON [ instance] ]
\end{verbatim}

The series repo config part of the configuration string is currently blank for all document repo types.

The second part of the configuration string defines the underlying implementation and its configuration. In
most cases the configuration associated with the implementation has a \verb+prefix+ parameter that is used to
define a table or a collection or a prefix for such entities in the underlying storage. The underlying implementation
defines what lower level software is used to host the data managed by \Rapture. The following table shows the current
implementations:

\begin{table}[h]
  \small
\begin{center}
\begin{tabular}{r l p{7cm}}
  Keyword & Underlying & Configuration \\
  \hline
  MONGODB & MongoDb & The prefix parameter defines the name of the collections used by this repository. To avoid
  conflict this is usually a function of the name of the \Rapture repository. \\
  CASSANDRA & Cassandra & The prefix parameter defines the name of the collections used by this repository. To avoid
  conflict this is usually a function of the name of the \Rapture repository. \\
\end{tabular}
\end{center}
\end{table}

The \verb+ON+ directive defines which configuration will be used to connect to the underlying store. If
not present the \verb+DEFAULT+ configuration will be used. These keywords are used by the underlying
implementation to load a system specific configuration file, environment variables or property set.

For example the default configuration for MongoDb (\verb+ON DEFAULT+) instructs the MongoDB implementation
to look in three places for a connection string to a MongoDB server -

\begin{itemize}
\item{The environment variable RAPTUREMONGODB-DEFAULT.}
\item{The java property RAPTUREMONGODB-DEFAULT.}
\item{The line beginning default= in the file RaptureMONGODB.cfg on the classpath of the application.}
\end{itemize}

In most cases the implementation will read the value from the file associated with the application server.

Using this technique multiple underlying servers can be used and repositories attached to them using the
\verb+ON+ keyword.
