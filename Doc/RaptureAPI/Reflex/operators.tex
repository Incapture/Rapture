\chapter{Operators}
Operators \index{operator} in \Reflex will, in the most part, be very familiar to developers of other languages. There are boolean \index{boolean} operators ($==$, $>=$, $<=$, $<$, $>$, $!=$, $||$, $\&\&$), arithmetic \index{arithmetic} operators ($+$,$-$,$/$,$*$,$\%$) and index \index{index} operators (\Verb+[ ]+). The ternary \index{ternary} operator $?$ is also supported. The use of these \emph{simple} operators is best illustrated by example. The examples below also introduce the \verb+assert+ built-in function - it aborts the \Reflex script with an error if the result of the boolean expression is \verb+false+.
\begin{lstlisting}[caption={Simple operators}]
// Examples of use of simple operators
// Boolean operators
assert(true);
assert(true || false);
assert(!false);
assert(true && true);
// Relational
assert(1 < 2);
assert(55 >= 55);
assert('a' < 'b'); // Note that strings can be compared
// Addition
assert(1 + 999 == 1000);
assert([1] + 1 == [1,1]); // Note addition on lists
assert([1,2,3] - 3 == [1,2]); // Note subtraction on lists
// Multiply
assert(3 * 50 == 150);
assert(4 / 2 == 2);
assert(999 % 3 == 0); // % = mod operator
// Power
assert(2 ^ 3 == 8);
\end{lstlisting}
It is worth calling out explicitly how $+$ and $-$ work with lists. If the left hand side of an expression is a list, then adding an element to it results in a new list with that element added to the end. Subtracting from a list removes that element from the list if it is within the list. This also works with strings.

The index \Verb+[ ]+ operator is worth its own set of examples.
\begin{lstlisting}[caption={Index operator}]
// Examples of the index operator
a = [1, 2, 3, 4, 5];
b = 'abcdefg';
assert(a[0] == 1);
assert(a[1 .. 2] == [2,3]);
assert(b[0] == 'a');
assert(b[1 .. 2] == 'bc');
\end{lstlisting}
There are two forms of the index operator. The first, with one integer parameter, simply returns the element at that position. The second, with the \Verb+..+ directive is a range operator - it returns the elements between these index points, inclusive of the first parameter and exclusive of the second.

The index operator also applies to map types as well. In this case the parameter is a string and refers to the key to lookup in the associative map.
\begin{lstlisting}[caption={Index operator on maps}]
a = { 'one' : 1, 'two' : 2 };
assert(a['one'] == 1);
assert(a['two'] == 2);
\end{lstlisting}

The dot operator ($.$) can be used with maps to reference elements of that map:

\begin{lstlisting}[caption={Dot operator on maps}]
a = { 'one' : 1, 'two' : 2 };
assert(a.one == 1);
assert(a.two == 2);
a.three = 3;
assert(a.three == 3);
\end{lstlisting}
