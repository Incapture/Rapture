The Index API is used in \Rapture to manage alternate ways to find data. Most of the
\Rapture repositories have a key/value approach to managing data - if you know the key (the uri) then
you can find the data. An index is the means by which you can
store additional information that you can query in a more relational way to find data in \Rapture.

An index in \Rapture is always associated with a document repository, and the index itself is stored
in the same underlying system as that repository. The uri of an index must match the uri of a document repository.

The configuration of an index is something that defines the columns that will form the index (both queryable values and those returned in a result set) and
the means for setting the values in that index. The data can either come from the URI of the written document or the content.

A typical example of a configuration string is reproduced below.

\begin{Verbatim}
   //docTest --> field1($0) string, field2(test) string
\end{Verbatim}

In this example the first entry in the index "field1" is of type "string" and is mapped to \verb+$0+ which
is a directive that means the first entry in the URI of the document (after the repository name).

The second entry in this index "field2" is also of type "string" but is mapped to the field value in the document written
with the key "test". A dotted separator convention is used to index more deeply nested fields. Numbers can be used for array values
to indicate a particular index of that array.

With this example, an attempt to write the following document:

\begin{Verbatim}
   //docTest/one/two

   {
      "test" : "hello"
   }
\end{Verbatim}

would result in a new entry in the index containing the following values:

\begin{table}[h]
  \small
\begin{center}
\begin{tabular}{r l l}
  Rowid & field1 & field2 \\
  \hline
  \verb+//docTest/one/two+ & one & hello \\
\end{tabular}
\end{center}
\end{table}

The index management is wired directly into the \verb+putContent+ implementation of the
document API. It also handles deletions and version modifications.
