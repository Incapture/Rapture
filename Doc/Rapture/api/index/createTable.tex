A table in \Rapture is simply a low level representation of a queryable table and is used
internally in \Rapture server applications. You cannot add data to a table through the \Rapture API.

The \verb+createTable+ call is used to define a table and its associated storage on \Rapture. The format
of the configuration string is very similar to that used by document, blob and series repositories. It is
best demonstrated with an example:

\begin{Verbatim}
TABLE {} USING MONGODB { prefix = 'test' }
\end{Verbatim}

The general form of the configuration is:

\begin{Verbatim}
TABLE { }
     USING [underlying implementation] { [ config ]}
     [ ON [ instance] ]
\end{Verbatim}

The second part of the configuration string defines the underlying implementation and its configuration. In
most cases the configuration associated with the implementation has a \verb+prefix+ parameter that is used to
define a table or a collection or a prefix for such entities in the underlying storage. The underlying implementation
defines what lower level software is used to host the data managed by \Rapture. The following table shows the current
implementations:

\begin{table}[h]
  \small
\begin{center}
\begin{tabular}{r l p{7cm}}
  Keyword & Underlying & Configuration \\
  \hline
  MONGODB & MongoDb & The prefix parameter defines the name of the collections used by this repository. To avoid
  conflict this is usually a function of the name of the \Rapture repository. \\
  CASSANDRA & Cassandra & The prefix parameter defines the name of the collections used by this repository. To avoid
  conflict this is usually a function of the name of the \Rapture repository. \\
  POSTGRES & PostgresSql &  The prefix parameter defines the name of the tables used by this repository. To avoid
  conflict this is usually a function of the name of the \Rapture repository. \\
\end{tabular}
\end{center}
\end{table}
