Plugins in \Rapture are installable items that can be released to a \Rapture environment in
a consistent way. There are three parts to the Plugin API.

The first part relates to retrieving information about what is actually installed, and what items were installed with each
plugin. These are used by operator consoles and are informational.

The second part relates to installing an actual plugin. These calls are typically used by a specific
client side application "PluginInstaller" or a self installing library that uses the "SelfPluginInstallerLib". Plugins
defined in this way are jar archives that have a very specific structure. Typically a developer lays out this structure
in the file system and then uses a build process to package this information up into a jar file which can then be used with
the PluginInstaller application.

\section{File layout of a plugin}

A sample layout of a plugin "on disk" is reproduced below:

\begin{Verbatim}
  -- src/main/resources
     - PLUGIN
        - plugin.txt
        - content
          - testRepo
             - .rdoc
             someContent.rdoc
          - myscripts
             - test.script
\end{Verbatim}

The mandatory aspects of a plugin are the concept of a PLUGIN folder in the resources within
which is a plugin.txt file. This file has a specific format, an example is reproduced below:

\begin{Verbatim}
  {
     "depends":{},
     "description":"Curtis Web Core",
     "plugin":"CurtisWebCore",
     "version":{
     		"major":1,
     		"minor":1,
     		"release":32,
     		"timestamp":99999999999999
     }
  }
\end{Verbatim}

Most of these fields are self describing - the "plugin" field is the unique reference (uri) of this
plugin and the version section defines the version \emph{of this plugin}. The description is for operators
and the depends field can contain a map of other plugins and their version numbers required.

\section{Plugin file types}

The content folder contains information and definitions for content and configuration that this plugin defines. The
file name extension defines what the plugin installer will do for this item.

\begin{table}[H]
\begin{center}
\begin{tabular}{r p{10cm}}
  Extension & Purpose \\
  \hline
  script & Defines a \Reflex script. The full uri of the script is formed from this file's location in the content directory.\\
  series & Defines set of points in a series (if fully named) or the definition of a series repository. In the latter case the .series file
     must be at the 2nd top level folder in the content folder - the name of the folder is the name of the repository in \Rapture. \\
  blob & Defines a blob in a blob repository or the definition of a blob repository (if the file is .blob). Note that files that are not in
    the list of extensions in this table are treated as blobs, with the mime type inferred from the extension of the file. \\
  rdoc & Defines a document in a document repository \emph{or} the definition of a document repository (if no name specified). \\
  idgen & Defines an idgenerator. The name of the generator is defined by the top level folder, the contents is the configuration of an ID generator.\\
  revent & Defines a rapture event. The content of the file is a RaptureEvent structure (as json). \\
  table & Defines a rapture table. The name of the table is the enclosing folder. \\
  job & Defines a schedule job. \\
  workflow & Defines a workflow. \\
  lock & Defines a lock manager. \\
  structured & Defines a structure store reference. \\
\end{tabular}
\end{center}
\end{table}

In all cases for these files where the content drives the definition of something in \Rapture, the content is
a json formatted document that is the equivalent of serializing the result of the appropriate "get" call for that entity.

For example, the document "getDocRepoConfig" call returns a "DocumentRepoConfig" which has a complex structure of
information about the description, document repo configuration and so on. In a file format this will look as follows:

\begin{Verbatim}
  {
    "authority":"curtis.menu",
    "strictCheck":true,
    "indexes":[],
    "fullTextIndexes":[],
    "documentRepo":
       {
          "authority":"curtis.menu",
          "config":"NREP {} USING MONGODB { prefix=\"curtis.menu\" }"
       }
  }
\end{Verbatim}

To obtain the format for this you could also use the following \Reflex code:

\begin{Verbatim}
   config = #document.getDocRepoConfig('//curtis.menu');
   println(json(config));
\end{Verbatim}

The same technique can be used with other entity types.

\section{Plugin Build Process}
To build a plugin that has the resource configuration reproduced above the following
\emph{gradle} build process is recommended:

\begin{Verbatim}[fontsize=\footnotesize]
  raptureVersion = "3.0.0"

  apply plugin: 'application'

  dependencies {
          compile "net.rapture:SelfPluginInstallerLib:$raptureVersion"
  }

  mainClassName = "rapture.plugin.app.SelfInstaller"

  task fatJar(type: Jar) {
       manifest {
          attributes 'Implementation-Title' : 'Plugin self installer',
                     'Implementation-Version' : '1.0.0',
                     'Main-Class' : mainClassName
      }
      baseName = project.name
      from {
          configurations.compile.collect {
               it.isDirectory() ? it : zipTree(it) }
      } with jar
  }

  fatJar.dependsOn(compileJava)

\end{Verbatim}

This (simplified) build script uses the \Rapture open source SelfInstaller library which contains
a main entry point that can be used to install the plugin into a \Rapture environment.

Running \verb+gradle fatJar+ will create a single jar that can be invoked with 'java -jar' along with
the following command line options to connect it to a \Rapture environment. The self installer will use
the Plugin API described here to install the components registered in this Plugin.

\begin{table}[H]
\begin{center}
\begin{tabular}{r l p{10cm}}
  Extension & Purpose \\
  \hline
  \verb+--host+ & The \Rapture host to connect to. \\
  \verb+--user+ & The \Rapture user to connect as. \\
  \verb+--password+ & The \Rapture password to use for that user. \\
  \verb+--overlay+ & What overlay to use (optional) (see below). \\
\end{tabular}
\end{center}
\end{table}

\section{Plugin overlays}
A plugin file can contain other sections in addition to the "content" folder described above. These are known
as overlays and this can be used to combine the "default" content folder with a given overlay. Overlays are used
to perhaps define slightly different configuration documents for a development environment to a production environment -- in this case
the documents describing production changes would be specified in a "prod" folder - the sub contents following the same format
in style as the content folder. If this folder exists then passing \verb+--overlay prod+ to an installer would instruct the installer
to apply the content folder first, and then overlay the prod folder onto that, creating a merged view of what should be
installed for this plugin on this environment.

\section{Creating plugins from existing content}
The Plugin API contains calls that can be used when a \Rapture environment has existing content and configuration that needs to be captured in plugin format.

The basic approach is to first create a manifest (\Verb+createManifest+) and then add items to that manifest using the \Verb+add+ calls.
The version of the plugin needs to be set using \Verb+setManifestVersion+ and then the complete plugin is exported using \Verb+exportPlugin+. The content generated
is then of a form that can be installed using the PluginInstaller tool on a different environment, or extracted to be copied into a more
traditional Plugin project described above.
