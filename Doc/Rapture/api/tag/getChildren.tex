The \verb+getChildren+ call is normally used by user interfaces that wish
to present a browser type interface on a document repository. The call returns all documents
and "sub folders" (to a given depth) below a given point in the hierarchy implied
by the naming conventions used in uris. Typically an interface will use \verb+/+ as
the initial prefix and then append onto that prefix the names of either tag descriptions
or folders for further \verb+getChildren+ calls or \verb+getTagDescription+ if the location
maps to a real document.

The \verb+RaptureFolderInfo+ structure returned by this call is described below:

\begin{table}[H]
  \small
\begin{center}
\begin{tabular}{r l p{8cm}}
  Field & Type & Description \\
  \hline
  name & String & The name of this element. \\
  folder & Boolean & Whether the name refers to a document or a sub-folder \\
\end{tabular}
\end{center}
\end{table}
