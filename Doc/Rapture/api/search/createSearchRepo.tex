The \verb+createSearchRepo+ is used to create a new search repository in \Rapture. The parameters to the call
look straightforward -- simply the name of the new repository and a configuration string. The configuration string is
in fact a complex instruction written in a search repository domain specific language (DSL) that is used to define the
capabilities and underlying implementation of the repository.

The typical configuration string for a versioned repository backed by MongoDB is reproduced below:

\begin{Verbatim}
SEARCH {} USING ELASTIC { index = 'test' }
\end{Verbatim}

The general form of the configuration is:

\begin{Verbatim}
SEARCH { [ search repo config] }
     USING [underlying implementation] { [ config ]}
     [ ON [ instance] ]
\end{Verbatim}

The search repo config part of the configuration string is currently blank for all search repo types.

The second part of the configuration string defines the underlying implementation and its configuration. In
most cases the configuration associated with the implementation has an \verb+index+ parameter that is used to
define a table or a collection or a prefix for such entities in the underlying storage. The underlying implementation
defines what lower level software is used to host the data managed by \Rapture. The following table shows the current
implementations:

\begin{table}[H]
\small
\begin{center}
\begin{tabular}{r l p{8cm}}
  Keyword & Underlying & Configuration \\
  \hline
  ELASTIC & ElasticSearch & The index parameter defines the name of the index used by this repository. \\
\end{tabular}
\end{center}
\end{table}

The \verb+ON+ directive defines which configuration will be used to connect to the underlying store. If
not present the \verb+DEFAULT+ configuration will be used. These keywords are used by the underlying
implementation to load a system specific configuration file, environment variables or property set.

For example the default configuration for MongoDb (\verb+ON DEFAULT+) instructs the MongoDB implementation
to look in three places for a connection string to a MongoDB server -

\begin{itemize}
\item{The environment variable RAPTUREES-DEFAULT.}
\item{The java property RAPTURES-DEFAULT.}
\item{The line beginning default= in the file RaptureES.cfg on the classpath of the application.}
\end{itemize}

In most cases the implementation will read the value from the file associated with the application server.

Using this technique multiple underlying servers can be used and repositories attached to them using the
\verb+ON+ keyword.

The ON configuration can also be set using a Sys Api call, viz:

\begin{lstlisting}
	#sys.putConnectionInfo("ES", fromjson('{
			"host" : "127.0.0.1",
			"port" : 9300,
			"username" : "rapture",
			"password" : "rapture",
			"dbName" : "default",
			"instanceName" : "default",
			"CLASS" : "rapture.common.ConnectionInfo"
			}'
			));
\end{lstlisting}
