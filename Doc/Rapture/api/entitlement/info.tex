The Entitlements API manages entitlements in \Rapture. All API calls in \Rapture can be guarded using
entitlements that are defined through this API. Each API call definition has an entitlement check string and
permissions to access that entitlement can be assigned to entitlement groups. Users can be assigned to entitlement groups.

Whenever an API call is made in \Rapture the user associated with that call is checked using the groups they belong to
and whether those groups have that entitlement permission.

Some entitlements on APIs are "parameterized". As an example the document API call \verb+getDoc+ has the entitlement path:

\begin{Verbatim}
  /data/read/$f(docUri)
\end{Verbatim}

The \verb+$f(docUri)+ part of that entitlement path is replaced with the \verb+docUri+ parameter passed to the call.
So a request to :

\begin{Verbatim}
    putDoc('//test/one/two', ...);
\end{Verbatim}

would result in an entitlement check for \verb+/data/read/test/one/two+.

The entitlement check code for this path works as follows:

\begin{enumerate}
  \item{check for an entitlement of \verb+/data/read/test/one/two+.}
  \item{check for an entitlement of \verb+/data/read/test/one+.}
  \item{check for an entitlement of \verb+/data/read/test+.}
  \item{check for an entitlement of \verb+/data/read+.}
  \item{check for an entitlement of \verb+/data+.}
\end{enumerate}

If any match the following check is performed and no other checks are made:

\begin{enumerate}
  \item{load entitlement}
  \item{for each group in entitlement, check for user membership}
\end{enumerate}

If the user is in any group that is in an entitlement the permission is granted. A null
entitlement (an empty) is normally defined as "all access granted".

The methods in the entitlement api are concerned with managing groups (sets of users) and entitlements (sets of groups).
