The Schedule API is used to manage the scheduling system of \Rapture. If a schedule server is running
in \Rapture it will look for upcoming jobs and ask other \Rapture servers to execute those jobs when they are due.

Jobs in \Rapture can either be scripts or workflows. Their schedule is defined by a cron type expression which can
be further controlled with a time zone parameter. If the job (according to its schedule) should be run and it is
activated a message is placed on the internal \Rapture pipeline for that job to be executed. A server in \Rapture
monitoring that pipeline (most \Rapture servers do this) can pick up this request and execute the job.

A job can also be marked as auto activate -- with this configuration the job will be initially set to be "activated" and therefore
will be run when its time based criteria is met. If the job is running it will not be rescheduled but once the execution
has completed the job will be activated once more and will run at its next scheduled time.

Jobs that are manually activated will not be scheduled until the activate job API call is called. This technique can be used to
schedule a job when some other external activity is valid or complete.

Jobs may not necessarily run at the exact time of their schedule -- the request to run the job is made at the time
scheduled but depending on the load of the system and the availability of resources the request may be queued behind other requests
and be delayed.

The API calls in this section have two purposes -- the first is in the definition and control of jobs, the second is a set
of UI convenience API calls that can be used to populate a calendar of jobs.
