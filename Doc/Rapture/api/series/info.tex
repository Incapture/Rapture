
The series API is used to manipulate the presence and the content of series repositories
in \Rapture.

In the abstract a series repository in \Rapture marries a key (a uri like other content in \Rapture) with
a set of points. Each point has a key (sometimes referred to as a column name) and a value. The value
can be a discrete type such as a long or a string. It can also be a more complex structure where the meaning
is known to the application developer or user of the repository.

In many cases the column name will be a string representation of a timestamp or point in time. Usually these
are formatted in such a way as to make lexical sorting of the columns imply a movement through time - for example
using a YYYYMMDD format. Lexically 20160401 is "after" 20160331 which is what the user usually wants from the
time series "specialization" of a \Rapture series repository.

Series repositories in \Rapture are backed by concrete data storage systems. When
you define a repository in \Rapture you provide a configuration string that is used
by \Rapture to route your request to a low level driver that interacts with the
underlying system. The format of this configuration string will be described in
the API call for creating a repository.

The URI of a series in a repository implies a folder-like structure with the
forward slash delineating these folders. There are series API calls to treat a
series repository like a file system -- these are useful when constructing
browsable user interfaces to a repository.

\subsection{Methods}
