The \verb+createBlobRepo+ is used to create a new blob repository in \Rapture. The two configuration strings
act in a similar way to the Document API's \verb+createDocRepo+ call, in fact the \verb+metaConfig+ parameter
is used to define an internal document repository that will be used to store and manage the meta data for the
repository. The documentation for \verb+createDocRepo+ shows the possible options for that configuration.

The main configuration string is a similar domain specific language that is used to pass configuration
options to an underlying implementation. The current options are simply described with an example.

The typical configuration string for a blob repository backed by MongoDB is reproduced below:

\begin{verbatim}
BLOB {} USING MONGODB { prefix = 'testBlob' }
\end{verbatim}

The general form of the configuration is:

\begin{verbatim}
BLOB { [ blob repo config] }
     USING [underlying implementation] { [ config ]}
     [ ON [ instance] ]
\end{verbatim}

The blob repo config part of the configuration string is currently blank for all blob repo types.

The second part of the configuration string defines the underlying implementation and its configuration. In
most cases the configuration associated with the implementation has a \verb+prefix+ parameter that is used to
define a table or a collection or a prefix for such entities in the underlying storage. The underlying implementation
defines what lower level software is used to host the data managed by \Rapture. The following table shows the current
implementations:

\begin{table}[h]
  \small
\begin{center}
\begin{tabular}{r l p{7cm}}
  Keyword & Underlying & Configuration \\
  \hline
  MONGODB & MongoDb & The prefix parameter defines the name of the collections used by this repository. To avoid
  conflict this is usually a function of the name of the \Rapture repository. \\
  CASSANDRA & Cassandra & The prefix parameter defines the name of the collections used by this repository. To avoid
  conflict this is usually a function of the name of the \Rapture repository. \\
\end{tabular}
\end{center}
\end{table}

There are some additional directives that can be used in a repository configuration definition.

The \verb+ON+ directive defines which configuration will be used to connect to the underlying store. If
not present the \verb+DEFAULT+ configuration will be used. These keywords are used by the underlying
implementation to load a system specific configuration file, environment variables or property set.

For example the default configuration for MongoDb (\verb+ON DEFAULT+) instructs the MongoDB implementation
to look in three places for a connection string to a MongoDB server -

\begin{itemize}
\item{The environment variable RAPTUREMONGODB-DEFAULT.}
\item{The java property RAPTUREMONGODB-DEFAULT.}
\item{The line beginning default= in the file RaptureMONGODB.cfg on the classpath of the application.}
\end{itemize}

In most cases the implementation will read the value from the file associated with the application server.

Using this technique multiple underlying servers can be used and repositories attached to them using the
\verb+ON+ keyword.
