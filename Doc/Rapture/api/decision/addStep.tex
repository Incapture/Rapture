The \verb+addStep+ call appends the passed in step to the workflow definition. This is
a call that is often used when defining a workflow for the first time, along with \verb+addTransition+.

The call \verb+putWorkflow+ is used to create the initial skeleton for the workflow.

The \verb+Step+ parameter is a complex structure that has the following fields.




\begin{table}[h]
  \small
\begin{center}
\begin{tabular}{r l p{7cm}}
  Field & Type & Description \\
  \hline
  name & string & The name of this step in the workflow. Used in transition references and the "startStep" field of a workflow.\\
  description & string & A description of this workflow. Used in user interfaces.\\
  executable & string & The task that will be executed as part of this step. (See below) \\
  view & map(string, string) & The view for this step -- the variables associated with this step. (See below).\\
  transitions & list(transitions) &  The transitions for this step. (See below).\\
  categoryOverride & string & If the step can only be executed on a certain group of \Rapture servers, this defines that group.\\
  softTimeout & integer & An indiciation as to how long (in seconds) the step is expected to run.\\
\end{tabular}
\end{center}
\end{table}

\subsubsection{Step Executable}
The step executable field of the \verb+Step+ type defines the underlying program that will be run
when this step is executed. It is a uri where the scheme can be one of the folllowing:

\begin{table}[h]
  \small
\begin{center}
\begin{tabular}{l p{6cm}}
  Scheme + Example & Description \\
  \hline
  \verb+script://test/one+ & Runs the \Reflex script referenced with the variables in the table below set. \\
  \verb+dp_java_invocable://myclass+ & Instantiate the java class \verb+rapture.dp.invocable.myclass+ which implements the \verb+AbstractInvocable+ interface. \\
  \verb+workflow://other/two+ & Spawns and switches to the attached workflow. The return value from the workflow is the return value of the step.\\
\end{tabular}
\end{center}
\end{table}

\begin{table}[h]
  \small
\begin{center}
\begin{tabular}{r p{10cm}}
  Variable Name & Meaning \\
  \hline
  \verb+$DP_WORK_ORDER_URI+ & The uri of the work order \\
  \verb+$DP_WORKER_URI+ & The uri of the worker \\
  \verb+$DP_WORKER_ID+ & The worker id \\
  \verb+$DP_STEP_NAME+ & The step name being executed \\
  \verb+$DP_STEP_START_TIME+ & The step start time \\
\end{tabular}
\end{center}
\end{table}
