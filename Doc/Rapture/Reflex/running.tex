\chapter{Running Reflex}
There are three options for using \Reflex. The most common use for \Reflex is to run scripts from within a \Rapture environment - you upload scripts to \Rapture and then call them through either \Rapture's API call \Verb+runScript+ or through a \Rapture workflow or event handling.

For testing and debugging it is preferable to install a local environment to play with. This section describes how to do that.

\Reflex is bundled into an application called \emph{ReflexRunner} that can be used to run \Reflex scripts. ReflexRunner is a command line java application that can be downloaded from the \href{http://incapture.github.com/RaptureRepo/release}{\Rapture Release Site}. Once downloaded it can be run using java as follows:

\begin{Verbatim}[fontsize=\small]
  java -jar ReflexRunner [options]

  ReflexRunner [-r Rapture API URL]
               [-u Rapture username]
               [-p Rapture password]
               [-f Script file]
               [-params Script parameters]
               [-d Debug]
               [-i Instrument]
\end{Verbatim}

\subsection{Examples}

Execute a script with a parameter:
\begin{Verbatim}
ReflexRunner -r http://localhost:8665/rapture
             -f myScript.rfx -param x=1
\end{Verbatim}
Execute a script with 2 parameters:
\begin{Verbatim}
ReflexRunner -r http://localhost:8665/rapture
             -f myScript.rfx -param x=1,y=abc
\end{Verbatim}
Execute a script with instrumentation:
\begin{Verbatim}
ReflexRunner -r http://localhost:8665/rapture
             -f myScript.rfx -i
\end{Verbatim}

\subsection{Usage notes}
\begin{itemize}
\item{-r and -f are mandatory parameters}
\item{User name and password will be prompted for if not set using switches or environment variables.}
\item{Cannot use -i and -d switches at the same time}
\item{Parameter input format is a comma seperated list of: <variable name>=<variable value}
\end{itemize}

\subsection{Supported environment variables}
\begin{itemize}
\item{RAPTURE\_HOST sets -r}
\item{RAPTURE\_USER sets -u}
\item{RAPTURE\_PASSWORD sets -p}
\end{itemize}

\subsection{Exit codes}
\begin{itemize}
\item{0: No parameters passed.}
\item{1: Rapture URL (-r) parameter or environment variable is missing.}
\item{2: Switches -d and -i cannot be used at same time.}
\end{itemize}
