\chapter{Background}
Recently there has been a significant move to create languages that run on top of the Java Virtual Machine.
These languages are usually integrated as a fundamental "Java Scripting Language" - such as JRuby \index{JRuby}
(an implementation of Ruby \index{Ruby}), Jython \index{Jython} (the equivalent of Python \index{Python}) and
Rhino \index{Rhino} (a JavaScript \index{JavaScript} implementation) or as a fully defined language that hooks
into the JVM at a lower level. Examples of these languages would include Clojure \index{Clojure} (a functional Lisp dialect),
Groovy \index{Groovy} (a scripting language) and Scala \index{Scala} (an object-oriented and functional programming language).

\Reflex is a procedural language that hooks into the JVM at a lower level. Its syntax is similar
to Python (without the indentation) and there are a number of built-in functions and special
operators that are semantic short cuts when interacting with \Rapture.

This part of the document provides a detailed description of \Reflex.
